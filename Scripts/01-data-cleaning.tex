% Options for packages loaded elsewhere
\PassOptionsToPackage{unicode}{hyperref}
\PassOptionsToPackage{hyphens}{url}
%
\documentclass[
]{article}
\usepackage{amsmath,amssymb}
\usepackage{iftex}
\ifPDFTeX
  \usepackage[T1]{fontenc}
  \usepackage[utf8]{inputenc}
  \usepackage{textcomp} % provide euro and other symbols
\else % if luatex or xetex
  \usepackage{unicode-math} % this also loads fontspec
  \defaultfontfeatures{Scale=MatchLowercase}
  \defaultfontfeatures[\rmfamily]{Ligatures=TeX,Scale=1}
\fi
\usepackage{lmodern}
\ifPDFTeX\else
  % xetex/luatex font selection
\fi
% Use upquote if available, for straight quotes in verbatim environments
\IfFileExists{upquote.sty}{\usepackage{upquote}}{}
\IfFileExists{microtype.sty}{% use microtype if available
  \usepackage[]{microtype}
  \UseMicrotypeSet[protrusion]{basicmath} % disable protrusion for tt fonts
}{}
\makeatletter
\@ifundefined{KOMAClassName}{% if non-KOMA class
  \IfFileExists{parskip.sty}{%
    \usepackage{parskip}
  }{% else
    \setlength{\parindent}{0pt}
    \setlength{\parskip}{6pt plus 2pt minus 1pt}}
}{% if KOMA class
  \KOMAoptions{parskip=half}}
\makeatother
\usepackage{xcolor}
\usepackage[margin=1in]{geometry}
\usepackage{color}
\usepackage{fancyvrb}
\newcommand{\VerbBar}{|}
\newcommand{\VERB}{\Verb[commandchars=\\\{\}]}
\DefineVerbatimEnvironment{Highlighting}{Verbatim}{commandchars=\\\{\}}
% Add ',fontsize=\small' for more characters per line
\usepackage{framed}
\definecolor{shadecolor}{RGB}{248,248,248}
\newenvironment{Shaded}{\begin{snugshade}}{\end{snugshade}}
\newcommand{\AlertTok}[1]{\textcolor[rgb]{0.94,0.16,0.16}{#1}}
\newcommand{\AnnotationTok}[1]{\textcolor[rgb]{0.56,0.35,0.01}{\textbf{\textit{#1}}}}
\newcommand{\AttributeTok}[1]{\textcolor[rgb]{0.13,0.29,0.53}{#1}}
\newcommand{\BaseNTok}[1]{\textcolor[rgb]{0.00,0.00,0.81}{#1}}
\newcommand{\BuiltInTok}[1]{#1}
\newcommand{\CharTok}[1]{\textcolor[rgb]{0.31,0.60,0.02}{#1}}
\newcommand{\CommentTok}[1]{\textcolor[rgb]{0.56,0.35,0.01}{\textit{#1}}}
\newcommand{\CommentVarTok}[1]{\textcolor[rgb]{0.56,0.35,0.01}{\textbf{\textit{#1}}}}
\newcommand{\ConstantTok}[1]{\textcolor[rgb]{0.56,0.35,0.01}{#1}}
\newcommand{\ControlFlowTok}[1]{\textcolor[rgb]{0.13,0.29,0.53}{\textbf{#1}}}
\newcommand{\DataTypeTok}[1]{\textcolor[rgb]{0.13,0.29,0.53}{#1}}
\newcommand{\DecValTok}[1]{\textcolor[rgb]{0.00,0.00,0.81}{#1}}
\newcommand{\DocumentationTok}[1]{\textcolor[rgb]{0.56,0.35,0.01}{\textbf{\textit{#1}}}}
\newcommand{\ErrorTok}[1]{\textcolor[rgb]{0.64,0.00,0.00}{\textbf{#1}}}
\newcommand{\ExtensionTok}[1]{#1}
\newcommand{\FloatTok}[1]{\textcolor[rgb]{0.00,0.00,0.81}{#1}}
\newcommand{\FunctionTok}[1]{\textcolor[rgb]{0.13,0.29,0.53}{\textbf{#1}}}
\newcommand{\ImportTok}[1]{#1}
\newcommand{\InformationTok}[1]{\textcolor[rgb]{0.56,0.35,0.01}{\textbf{\textit{#1}}}}
\newcommand{\KeywordTok}[1]{\textcolor[rgb]{0.13,0.29,0.53}{\textbf{#1}}}
\newcommand{\NormalTok}[1]{#1}
\newcommand{\OperatorTok}[1]{\textcolor[rgb]{0.81,0.36,0.00}{\textbf{#1}}}
\newcommand{\OtherTok}[1]{\textcolor[rgb]{0.56,0.35,0.01}{#1}}
\newcommand{\PreprocessorTok}[1]{\textcolor[rgb]{0.56,0.35,0.01}{\textit{#1}}}
\newcommand{\RegionMarkerTok}[1]{#1}
\newcommand{\SpecialCharTok}[1]{\textcolor[rgb]{0.81,0.36,0.00}{\textbf{#1}}}
\newcommand{\SpecialStringTok}[1]{\textcolor[rgb]{0.31,0.60,0.02}{#1}}
\newcommand{\StringTok}[1]{\textcolor[rgb]{0.31,0.60,0.02}{#1}}
\newcommand{\VariableTok}[1]{\textcolor[rgb]{0.00,0.00,0.00}{#1}}
\newcommand{\VerbatimStringTok}[1]{\textcolor[rgb]{0.31,0.60,0.02}{#1}}
\newcommand{\WarningTok}[1]{\textcolor[rgb]{0.56,0.35,0.01}{\textbf{\textit{#1}}}}
\usepackage{graphicx}
\makeatletter
\def\maxwidth{\ifdim\Gin@nat@width>\linewidth\linewidth\else\Gin@nat@width\fi}
\def\maxheight{\ifdim\Gin@nat@height>\textheight\textheight\else\Gin@nat@height\fi}
\makeatother
% Scale images if necessary, so that they will not overflow the page
% margins by default, and it is still possible to overwrite the defaults
% using explicit options in \includegraphics[width, height, ...]{}
\setkeys{Gin}{width=\maxwidth,height=\maxheight,keepaspectratio}
% Set default figure placement to htbp
\makeatletter
\def\fps@figure{htbp}
\makeatother
\setlength{\emergencystretch}{3em} % prevent overfull lines
\providecommand{\tightlist}{%
  \setlength{\itemsep}{0pt}\setlength{\parskip}{0pt}}
\setcounter{secnumdepth}{5}
% definitions for citeproc citations
\NewDocumentCommand\citeproctext{}{}
\NewDocumentCommand\citeproc{mm}{%
  \begingroup\def\citeproctext{#2}\cite{#1}\endgroup}
\makeatletter
 % allow citations to break across lines
 \let\@cite@ofmt\@firstofone
 % avoid brackets around text for \cite:
 \def\@biblabel#1{}
 \def\@cite#1#2{{#1\if@tempswa , #2\fi}}
\makeatother
\newlength{\cslhangindent}
\setlength{\cslhangindent}{1.5em}
\newlength{\csllabelwidth}
\setlength{\csllabelwidth}{3em}
\newenvironment{CSLReferences}[2] % #1 hanging-indent, #2 entry-spacing
 {\begin{list}{}{%
  \setlength{\itemindent}{0pt}
  \setlength{\leftmargin}{0pt}
  \setlength{\parsep}{0pt}
  % turn on hanging indent if param 1 is 1
  \ifodd #1
   \setlength{\leftmargin}{\cslhangindent}
   \setlength{\itemindent}{-1\cslhangindent}
  \fi
  % set entry spacing
  \setlength{\itemsep}{#2\baselineskip}}}
 {\end{list}}
\usepackage{calc}
\newcommand{\CSLBlock}[1]{\hfill\break\parbox[t]{\linewidth}{\strut\ignorespaces#1\strut}}
\newcommand{\CSLLeftMargin}[1]{\parbox[t]{\csllabelwidth}{\strut#1\strut}}
\newcommand{\CSLRightInline}[1]{\parbox[t]{\linewidth - \csllabelwidth}{\strut#1\strut}}
\newcommand{\CSLIndent}[1]{\hspace{\cslhangindent}#1}
\usepackage{booktabs}
\usepackage{longtable}
\usepackage{array}
\usepackage{multirow}
\usepackage{wrapfig}
\usepackage{float}
\usepackage{colortbl}
\usepackage{pdflscape}
\usepackage{tabu}
\usepackage{threeparttable}
\usepackage{threeparttablex}
\usepackage[normalem]{ulem}
\usepackage{makecell}
\usepackage{xcolor}
\ifLuaTeX
  \usepackage{selnolig}  % disable illegal ligatures
\fi
\usepackage{bookmark}
\IfFileExists{xurl.sty}{\usepackage{xurl}}{} % add URL line breaks if available
\urlstyle{same}
\hypersetup{
  pdftitle={Data Cleaning},
  pdfauthor={Robert J. Dellinger},
  hidelinks,
  pdfcreator={LaTeX via pandoc}}

\title{Data Cleaning}
\author{Robert J. Dellinger}
\date{April 05, 2025}

\begin{document}
\maketitle

{
\setcounter{tocdepth}{2}
\tableofcontents
}
\section{Introduction}\label{introduction}

This document outlines the methodology for data cleaning, exploration,
and visualization. It is structured to ensure transparency and
reproducibility of all analyses.

\subsection{Methodology}\label{methodology}

Briefly describe the methods used in the project, including data
sources, cleaning steps, and techniques applied to handle missing or
inconsistent data.

\subsection{Loading Data}\label{loading-data}

The first step in any data analysis is to load the data. This section
outlines the process of importing the data into R, including any
necessary transformations or adjustments to ensure compatibility with
the analysis.

\begin{Shaded}
\begin{Highlighting}[]
\CommentTok{\# Example of loading data from a CSV file}
\CommentTok{\# raw\_data \textless{}{-} read\_csv(here("Data", "Raw", "data\_file.csv"))}
\CommentTok{\# raw\_data \textless{}{-} read\_excel(here("Data", "Raw", "data\_file.xlsx"))}
\end{Highlighting}
\end{Shaded}

\subsection{Cleaning Data}\label{cleaning-data}

The data cleaning process involves several steps to ensure the data is
in a suitable format for analysis. This includes handling missing
values, correcting data types, and removing duplicates.

\begin{Shaded}
\begin{Highlighting}[]
\CommentTok{\# Example of cleaning data}
\CommentTok{\# {-} Removing duplicates}
\CommentTok{\# {-} Handling missing values}
\CommentTok{\# {-} Converting data types and cleaning white space}
\CommentTok{\# {-} Renaming columns, etc.}

\CommentTok{\# cleaned\_data \textless{}{-} raw\_data \%\textgreater{}\%}
\CommentTok{\#   clean\_names() \%\textgreater{}\%}
\CommentTok{\#   mutate(column\_name = as\_factor(column\_name)) \%\textgreater{}\% }
\CommentTok{\#   mutate(date\_column = as.Date(date\_column, format = "\%Y{-}\%m{-}\%d")) \%\textgreater{}\% \# convert to date}
\CommentTok{\#   mutate(numeric\_column = as.numeric(numeric\_column)) \%\textgreater{}\% \# convert to numeric}
\CommentTok{\#   mutate(accross(everything(), \textasciitilde{}str\_squish(.))) \%\textgreater{}\% \# clean whitespace}
\CommentTok{\#   drop\_na()}
\end{Highlighting}
\end{Shaded}

\subsection{Data Exploration}\label{data-exploration}

Data exploration is a crucial step in understanding the dataset and
identifying patterns or anomalies. This section includes summary
statistics, visualizations, and any other relevant analyses to gain
insights into the data.

\begin{Shaded}
\begin{Highlighting}[]
\CommentTok{\# Explore the cleaned data using basic summaries:}
\CommentTok{\# glimpse(cleaned\_data)}
\CommentTok{\# summary(cleaned\_data)}
\CommentTok{\# str(cleaned\_data)}
\end{Highlighting}
\end{Shaded}

\subsection{Data Visualization}\label{data-visualization}

Data visualization is an essential part of data analysis, allowing for
the communication of findings in a clear and effective manner. This
section includes various plots and charts to illustrate key insights
from the data.

\begin{Shaded}
\begin{Highlighting}[]
\CommentTok{\# Example of creating a summary table }
\CommentTok{\# summary\_table \textless{}{-} cleaned\_data \%\textgreater{}\%}
\CommentTok{\#   group\_by(group\_var) \%\textgreater{}\%}
\CommentTok{\#   summarise(mean\_value = mean(value\_var, na.rm = TRUE)) \%\textgreater{}\%}
\CommentTok{\#   ungroup() \%\textgreater{}\%}
\CommentTok{\#   kable() \%\textgreater{}\%}
\CommentTok{\#   kable\_styling(full\_width = F, position = "left") }

\CommentTok{\# save\_kable(summary\_table, file = here(output\_path\_tables, "summary\_table.html"),}
\CommentTok{\#   bootstrap\_options = c("striped", "hover", "condensed"),}
\CommentTok{\#   full\_width = F, position = "left")}
\end{Highlighting}
\end{Shaded}

\begin{Shaded}
\begin{Highlighting}[]
\CommentTok{\# Example of visualization plot}
\CommentTok{\# ggplot(cleaned\_data, aes(x = var1, y = var2)) +}
\CommentTok{\#   geom\_point() +}
\CommentTok{\#   theme\_minimal()}

\CommentTok{\# Example of saving a plot}
\CommentTok{\# ggsave(filename = here(output\_path\_images, "plot\_name.png"), plot = last\_plot(), width = 6, height = 4)}
\end{Highlighting}
\end{Shaded}

\section{Literature Cited}\label{literature-cited}

Citing the packages and data used in the analysis is important for
reproducibility and transparency. The following code generates a
bibliography of all loaded packages. Items can be cited directly within
the documentation using the syntax \texttt{@key} where key is the
citation key in the first line of the entry, e.g., R Core Team (2024).
To put citations in parentheses, use \texttt{{[}@key{]}} instead.

(\textbf{ggplot2?})

\phantomsection\label{refs}
\begin{CSLReferences}{1}{0}
\bibitem[\citeproctext]{ref-R-base}
R Core Team. 2024. \emph{R: A Language and Environment for Statistical
Computing}. Vienna, Austria: R Foundation for Statistical Computing.
\url{https://www.R-project.org/}.

\end{CSLReferences}

\end{document}
