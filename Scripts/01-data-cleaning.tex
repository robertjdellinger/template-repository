% Options for packages loaded elsewhere
\PassOptionsToPackage{unicode}{hyperref}
\PassOptionsToPackage{hyphens}{url}
%
\documentclass[
]{article}
\usepackage{amsmath,amssymb}
\usepackage{iftex}
\ifPDFTeX
  \usepackage[T1]{fontenc}
  \usepackage[utf8]{inputenc}
  \usepackage{textcomp} % provide euro and other symbols
\else % if luatex or xetex
  \usepackage{unicode-math} % this also loads fontspec
  \defaultfontfeatures{Scale=MatchLowercase}
  \defaultfontfeatures[\rmfamily]{Ligatures=TeX,Scale=1}
\fi
\usepackage{lmodern}
\ifPDFTeX\else
  % xetex/luatex font selection
\fi
% Use upquote if available, for straight quotes in verbatim environments
\IfFileExists{upquote.sty}{\usepackage{upquote}}{}
\IfFileExists{microtype.sty}{% use microtype if available
  \usepackage[]{microtype}
  \UseMicrotypeSet[protrusion]{basicmath} % disable protrusion for tt fonts
}{}
\makeatletter
\@ifundefined{KOMAClassName}{% if non-KOMA class
  \IfFileExists{parskip.sty}{%
    \usepackage{parskip}
  }{% else
    \setlength{\parindent}{0pt}
    \setlength{\parskip}{6pt plus 2pt minus 1pt}}
}{% if KOMA class
  \KOMAoptions{parskip=half}}
\makeatother
\usepackage{xcolor}
\usepackage[margin=1in]{geometry}
\usepackage{color}
\usepackage{fancyvrb}
\newcommand{\VerbBar}{|}
\newcommand{\VERB}{\Verb[commandchars=\\\{\}]}
\DefineVerbatimEnvironment{Highlighting}{Verbatim}{commandchars=\\\{\}}
% Add ',fontsize=\small' for more characters per line
\usepackage{framed}
\definecolor{shadecolor}{RGB}{248,248,248}
\newenvironment{Shaded}{\begin{snugshade}}{\end{snugshade}}
\newcommand{\AlertTok}[1]{\textcolor[rgb]{0.94,0.16,0.16}{#1}}
\newcommand{\AnnotationTok}[1]{\textcolor[rgb]{0.56,0.35,0.01}{\textbf{\textit{#1}}}}
\newcommand{\AttributeTok}[1]{\textcolor[rgb]{0.13,0.29,0.53}{#1}}
\newcommand{\BaseNTok}[1]{\textcolor[rgb]{0.00,0.00,0.81}{#1}}
\newcommand{\BuiltInTok}[1]{#1}
\newcommand{\CharTok}[1]{\textcolor[rgb]{0.31,0.60,0.02}{#1}}
\newcommand{\CommentTok}[1]{\textcolor[rgb]{0.56,0.35,0.01}{\textit{#1}}}
\newcommand{\CommentVarTok}[1]{\textcolor[rgb]{0.56,0.35,0.01}{\textbf{\textit{#1}}}}
\newcommand{\ConstantTok}[1]{\textcolor[rgb]{0.56,0.35,0.01}{#1}}
\newcommand{\ControlFlowTok}[1]{\textcolor[rgb]{0.13,0.29,0.53}{\textbf{#1}}}
\newcommand{\DataTypeTok}[1]{\textcolor[rgb]{0.13,0.29,0.53}{#1}}
\newcommand{\DecValTok}[1]{\textcolor[rgb]{0.00,0.00,0.81}{#1}}
\newcommand{\DocumentationTok}[1]{\textcolor[rgb]{0.56,0.35,0.01}{\textbf{\textit{#1}}}}
\newcommand{\ErrorTok}[1]{\textcolor[rgb]{0.64,0.00,0.00}{\textbf{#1}}}
\newcommand{\ExtensionTok}[1]{#1}
\newcommand{\FloatTok}[1]{\textcolor[rgb]{0.00,0.00,0.81}{#1}}
\newcommand{\FunctionTok}[1]{\textcolor[rgb]{0.13,0.29,0.53}{\textbf{#1}}}
\newcommand{\ImportTok}[1]{#1}
\newcommand{\InformationTok}[1]{\textcolor[rgb]{0.56,0.35,0.01}{\textbf{\textit{#1}}}}
\newcommand{\KeywordTok}[1]{\textcolor[rgb]{0.13,0.29,0.53}{\textbf{#1}}}
\newcommand{\NormalTok}[1]{#1}
\newcommand{\OperatorTok}[1]{\textcolor[rgb]{0.81,0.36,0.00}{\textbf{#1}}}
\newcommand{\OtherTok}[1]{\textcolor[rgb]{0.56,0.35,0.01}{#1}}
\newcommand{\PreprocessorTok}[1]{\textcolor[rgb]{0.56,0.35,0.01}{\textit{#1}}}
\newcommand{\RegionMarkerTok}[1]{#1}
\newcommand{\SpecialCharTok}[1]{\textcolor[rgb]{0.81,0.36,0.00}{\textbf{#1}}}
\newcommand{\SpecialStringTok}[1]{\textcolor[rgb]{0.31,0.60,0.02}{#1}}
\newcommand{\StringTok}[1]{\textcolor[rgb]{0.31,0.60,0.02}{#1}}
\newcommand{\VariableTok}[1]{\textcolor[rgb]{0.00,0.00,0.00}{#1}}
\newcommand{\VerbatimStringTok}[1]{\textcolor[rgb]{0.31,0.60,0.02}{#1}}
\newcommand{\WarningTok}[1]{\textcolor[rgb]{0.56,0.35,0.01}{\textbf{\textit{#1}}}}
\usepackage{graphicx}
\makeatletter
\def\maxwidth{\ifdim\Gin@nat@width>\linewidth\linewidth\else\Gin@nat@width\fi}
\def\maxheight{\ifdim\Gin@nat@height>\textheight\textheight\else\Gin@nat@height\fi}
\makeatother
% Scale images if necessary, so that they will not overflow the page
% margins by default, and it is still possible to overwrite the defaults
% using explicit options in \includegraphics[width, height, ...]{}
\setkeys{Gin}{width=\maxwidth,height=\maxheight,keepaspectratio}
% Set default figure placement to htbp
\makeatletter
\def\fps@figure{htbp}
\makeatother
\setlength{\emergencystretch}{3em} % prevent overfull lines
\providecommand{\tightlist}{%
  \setlength{\itemsep}{0pt}\setlength{\parskip}{0pt}}
\setcounter{secnumdepth}{5}
% definitions for citeproc citations
\NewDocumentCommand\citeproctext{}{}
\NewDocumentCommand\citeproc{mm}{%
  \begingroup\def\citeproctext{#2}\cite{#1}\endgroup}
\makeatletter
 % allow citations to break across lines
 \let\@cite@ofmt\@firstofone
 % avoid brackets around text for \cite:
 \def\@biblabel#1{}
 \def\@cite#1#2{{#1\if@tempswa , #2\fi}}
\makeatother
\newlength{\cslhangindent}
\setlength{\cslhangindent}{1.5em}
\newlength{\csllabelwidth}
\setlength{\csllabelwidth}{3em}
\newenvironment{CSLReferences}[2] % #1 hanging-indent, #2 entry-spacing
 {\begin{list}{}{%
  \setlength{\itemindent}{0pt}
  \setlength{\leftmargin}{0pt}
  \setlength{\parsep}{0pt}
  % turn on hanging indent if param 1 is 1
  \ifodd #1
   \setlength{\leftmargin}{\cslhangindent}
   \setlength{\itemindent}{-1\cslhangindent}
  \fi
  % set entry spacing
  \setlength{\itemsep}{#2\baselineskip}}}
 {\end{list}}
\usepackage{calc}
\newcommand{\CSLBlock}[1]{\hfill\break\parbox[t]{\linewidth}{\strut\ignorespaces#1\strut}}
\newcommand{\CSLLeftMargin}[1]{\parbox[t]{\csllabelwidth}{\strut#1\strut}}
\newcommand{\CSLRightInline}[1]{\parbox[t]{\linewidth - \csllabelwidth}{\strut#1\strut}}
\newcommand{\CSLIndent}[1]{\hspace{\cslhangindent}#1}
\usepackage{booktabs}
\usepackage{longtable}
\usepackage{array}
\usepackage{multirow}
\usepackage{wrapfig}
\usepackage{float}
\usepackage{colortbl}
\usepackage{pdflscape}
\usepackage{tabu}
\usepackage{threeparttable}
\usepackage{threeparttablex}
\usepackage[normalem]{ulem}
\usepackage{makecell}
\usepackage{xcolor}
\ifLuaTeX
  \usepackage{selnolig}  % disable illegal ligatures
\fi
\usepackage{bookmark}
\IfFileExists{xurl.sty}{\usepackage{xurl}}{} % add URL line breaks if available
\urlstyle{same}
\hypersetup{
  pdftitle={Data Cleaning},
  pdfauthor={Robert J. Dellinger},
  hidelinks,
  pdfcreator={LaTeX via pandoc}}

\title{Data Cleaning}
\author{Robert J. Dellinger}
\date{April 05, 2025}

\begin{document}
\maketitle

{
\setcounter{tocdepth}{2}
\tableofcontents
}
\section{Introduction}\label{introduction}

This document outlines the methodology for data cleaning, exploration,
and visualization. It is structured to ensure transparency and
reproducibility of all analyses.

\subsection{Methodology}\label{methodology}

Briefly describe the methods used in the project, including data
sources, cleaning steps, and techniques applied to handle missing or
inconsistent data.

\subsection{Loading Data}\label{loading-data}

The first step in any data analysis is to load the data. This section
outlines the process of importing the data into R, including any
necessary transformations or adjustments to ensure compatibility with
the analysis.

\begin{Shaded}
\begin{Highlighting}[]
\CommentTok{\# Example of loading data from a CSV file}
\CommentTok{\# raw\_data \textless{}{-} read\_csv(here("Data", "Raw", "data\_file.csv"))}
\CommentTok{\# raw\_data \textless{}{-} read\_excel(here("Data", "Raw", "data\_file.xlsx"))}
\end{Highlighting}
\end{Shaded}

\subsection{Cleaning Data}\label{cleaning-data}

The data cleaning process involves several steps to ensure the data is
in a suitable format for analysis. This includes handling missing
values, correcting data types, and removing duplicates.

\begin{Shaded}
\begin{Highlighting}[]
\CommentTok{\# Example of cleaning data}
\CommentTok{\# {-} Removing duplicates}
\CommentTok{\# {-} Handling missing values}
\CommentTok{\# {-} Converting data types and cleaning white space}
\CommentTok{\# {-} Renaming columns, etc.}

\CommentTok{\# cleaned\_data \textless{}{-} raw\_data \%\textgreater{}\%}
\CommentTok{\#   clean\_names() \%\textgreater{}\%}
\CommentTok{\#   mutate(column\_name = as\_factor(column\_name)) \%\textgreater{}\% }
\CommentTok{\#   mutate(date\_column = as.Date(date\_column, format = "\%Y{-}\%m{-}\%d")) \%\textgreater{}\% \# convert to date}
\CommentTok{\#   mutate(numeric\_column = as.numeric(numeric\_column)) \%\textgreater{}\% \# convert to numeric}
\CommentTok{\#   mutate(accross(everything(), \textasciitilde{}str\_squish(.))) \%\textgreater{}\% \# clean whitespace}
\CommentTok{\#   drop\_na()}
\end{Highlighting}
\end{Shaded}

\subsection{Data Exploration}\label{data-exploration}

Data exploration is a crucial step in understanding the dataset and
identifying patterns or anomalies. This section includes summary
statistics, visualizations, and any other relevant analyses to gain
insights into the data.

\begin{Shaded}
\begin{Highlighting}[]
\CommentTok{\# Explore the cleaned data using basic summaries:}
\CommentTok{\# glimpse(cleaned\_data)}
\CommentTok{\# summary(cleaned\_data)}
\CommentTok{\# str(cleaned\_data)}
\end{Highlighting}
\end{Shaded}

\subsection{Data Visualization}\label{data-visualization}

Data visualization is an essential part of data analysis, allowing for
the communication of findings in a clear and effective manner. This
section includes various plots and charts to illustrate key insights
from the data.

\begin{Shaded}
\begin{Highlighting}[]
\CommentTok{\# Example of creating a summary table }
\CommentTok{\# summary\_table \textless{}{-} cleaned\_data \%\textgreater{}\%}
\CommentTok{\#   group\_by(group\_var) \%\textgreater{}\%}
\CommentTok{\#   summarise(mean\_value = mean(value\_var, na.rm = TRUE)) \%\textgreater{}\%}
\CommentTok{\#   ungroup() \%\textgreater{}\%}
\CommentTok{\#   kable() \%\textgreater{}\%}
\CommentTok{\#   kable\_styling(full\_width = F, position = "left") }

\CommentTok{\# save\_kable(summary\_table, file = here(output\_path\_tables, "summary\_table.html"),}
\CommentTok{\#   bootstrap\_options = c("striped", "hover", "condensed"),}
\CommentTok{\#   full\_width = F, position = "left")}
\end{Highlighting}
\end{Shaded}

\begin{Shaded}
\begin{Highlighting}[]
\CommentTok{\# Example of visualization plot}
\CommentTok{\# ggplot(cleaned\_data, aes(x = var1, y = var2)) +}
\CommentTok{\#   geom\_point() +}
\CommentTok{\#   theme\_minimal()}

\CommentTok{\# Example of saving a plot}
\CommentTok{\# ggsave(filename = here(output\_path\_images, "plot\_name.png"), plot = last\_plot(), width = 6, height = 4)}
\end{Highlighting}
\end{Shaded}

\section{Literature Cited}\label{literature-cited}

Citing the packages and data used in the analysis is important for
reproducibility and transparency. The following code generates a
bibliography of all loaded packages. Items can be cited directly within
the documentation using the syntax \texttt{@key} where key is the
citation key in the first line of the entry, e.g., R Core Team (2024).
To put citations in parentheses, use \texttt{{[}@key{]}} instead.

nocite: `'

\phantomsection\label{refs}
\begin{CSLReferences}{1}{0}
\bibitem[\citeproctext]{ref-R-rmarkdown}
Allaire, JJ, Yihui Xie, Christophe Dervieux, Jonathan McPherson, Javier
Luraschi, Kevin Ushey, Aron Atkins, et al. 2024. \emph{Rmarkdown:
Dynamic Documents for r}. \url{https://github.com/rstudio/rmarkdown}.

\bibitem[\citeproctext]{ref-sp2013}
Bivand, Roger S., Edzer Pebesma, and Virgilio Gomez-Rubio. 2013.
\emph{Applied Spatial Data Analysis with {R}, Second Edition}. Springer,
NY. \url{https://asdar-book.org/}.

\bibitem[\citeproctext]{ref-R-knitcitations}
Boettiger, Carl. 2021. \emph{Knitcitations: Citations for Knitr Markdown
Files}. \url{https://github.com/cboettig/knitcitations}.

\bibitem[\citeproctext]{ref-R-remotes}
Csárdi, Gábor, Jim Hester, Hadley Wickham, Winston Chang, Martin Morgan,
and Dan Tenenbaum. 2024. \emph{Remotes: R Package Installation from
Remote Repositories, Including GitHub}. \url{https://remotes.r-lib.org}.

\bibitem[\citeproctext]{ref-R-janitor}
Firke, Sam. 2024. \emph{Janitor: Simple Tools for Examining and Cleaning
Dirty Data}. \url{https://github.com/sfirke/janitor}.

\bibitem[\citeproctext]{ref-lubridate2011}
Grolemund, Garrett, and Hadley Wickham. 2011. {``Dates and Times Made
Easy with {lubridate}.''} \emph{Journal of Statistical Software} 40 (3):
1--25. \url{https://www.jstatsoft.org/v40/i03/}.

\bibitem[\citeproctext]{ref-R-glue}
Hester, Jim, and Jennifer Bryan. 2024. \emph{Glue: Interpreted String
Literals}. \url{https://glue.tidyverse.org/}.

\bibitem[\citeproctext]{ref-R-raster}
Hijmans, Robert J. 2025. \emph{Raster: Geographic Data Analysis and
Modeling}. \url{https://rspatial.org/raster}.

\bibitem[\citeproctext]{ref-ggmap2013}
Kahle, David, and Hadley Wickham. 2013. {``Ggmap: Spatial Visualization
with Ggplot2.''} \emph{The R Journal} 5 (1): 144--61.
\url{https://journal.r-project.org/archive/2013-1/kahle-wickham.pdf}.

\bibitem[\citeproctext]{ref-R-ggmap}
Kahle, David, Hadley Wickham, and Scott Jackson. 2023. \emph{Ggmap:
Spatial Visualization with Ggplot2}.
\url{https://github.com/dkahle/ggmap}.

\bibitem[\citeproctext]{ref-R-here}
Müller, Kirill. 2020. \emph{Here: A Simpler Way to Find Your Files}.
\url{https://here.r-lib.org/}.

\bibitem[\citeproctext]{ref-R-tibble}
Müller, Kirill, and Hadley Wickham. 2023. \emph{Tibble: Simple Data
Frames}. \url{https://tibble.tidyverse.org/}.

\bibitem[\citeproctext]{ref-sf2018}
Pebesma, Edzer. 2018. {``{Simple Features for R: Standardized Support
for Spatial Vector Data}.''} \emph{{The R Journal}} 10 (1): 439--46.
\url{https://doi.org/10.32614/RJ-2018-009}.

\bibitem[\citeproctext]{ref-R-sf}
---------. 2025. \emph{Sf: Simple Features for r}.
\url{https://r-spatial.github.io/sf/}.

\bibitem[\citeproctext]{ref-sp2005}
Pebesma, Edzer J., and Roger Bivand. 2005. {``Classes and Methods for
Spatial Data in {R}.''} \emph{R News} 5 (2): 9--13.
\url{https://CRAN.R-project.org/doc/Rnews/}.

\bibitem[\citeproctext]{ref-sf2023}
Pebesma, Edzer, and Roger Bivand. 2023. \emph{{Spatial Data Science:
With applications in R}}. {Chapman and Hall/CRC}.
\url{https://doi.org/10.1201/9780429459016}.

\bibitem[\citeproctext]{ref-R-sp}
---------. 2025. \emph{Sp: Classes and Methods for Spatial Data}.
\url{https://github.com/edzer/sp/}.

\bibitem[\citeproctext]{ref-R-patchwork}
Pedersen, Thomas Lin. 2024. \emph{Patchwork: The Composer of Plots}.
\url{https://patchwork.data-imaginist.com}.

\bibitem[\citeproctext]{ref-R-gganimate}
Pedersen, Thomas Lin, and David Robinson. 2025. \emph{Gganimate: A
Grammar of Animated Graphics}. \url{https://gganimate.com}.

\bibitem[\citeproctext]{ref-R-base}
R Core Team. 2024. \emph{R: A Language and Environment for Statistical
Computing}. Vienna, Austria: R Foundation for Statistical Computing.
\url{https://www.R-project.org/}.

\bibitem[\citeproctext]{ref-pacman2018}
Rinker, Tyler W., and Dason Kurkiewicz. 2018. \emph{{pacman}: {P}ackage
Management for {R}}. Buffalo, New York.
\url{http://github.com/trinker/pacman}.

\bibitem[\citeproctext]{ref-R-pacman}
Rinker, Tyler, and Dason Kurkiewicz. 2019. \emph{Pacman: Package
Management Tool}. \url{https://github.com/trinker/pacman}.

\bibitem[\citeproctext]{ref-plotly2020}
Sievert, Carson. 2020. \emph{Interactive Web-Based Data Visualization
with r, Plotly, and Shiny}. Chapman; Hall/CRC.
\url{https://plotly-r.com}.

\bibitem[\citeproctext]{ref-R-plotly}
Sievert, Carson, Chris Parmer, Toby Hocking, Scott Chamberlain, Karthik
Ram, Marianne Corvellec, and Pedro Despouy. 2024. \emph{Plotly: Create
Interactive Web Graphics via Plotly.js}. \url{https://plotly-r.com}.

\bibitem[\citeproctext]{ref-R-ggrepel}
Slowikowski, Kamil. 2024. \emph{Ggrepel: Automatically Position
Non-Overlapping Text Labels with Ggplot2}.
\url{https://ggrepel.slowkow.com/}.

\bibitem[\citeproctext]{ref-R-lubridate}
Spinu, Vitalie, Garrett Grolemund, and Hadley Wickham. 2024.
\emph{Lubridate: Make Dealing with Dates a Little Easier}.
\url{https://lubridate.tidyverse.org}.

\bibitem[\citeproctext]{ref-R-tigris}
Walker, Kyle. 2024. \emph{Tigris: Load Census TIGER/Line Shapefiles}.
\url{https://github.com/walkerke/tigris}.

\bibitem[\citeproctext]{ref-ggplot22016}
Wickham, Hadley. 2016. \emph{Ggplot2: Elegant Graphics for Data
Analysis}. Springer-Verlag New York.
\url{https://ggplot2.tidyverse.org}.

\bibitem[\citeproctext]{ref-R-forcats}
---------. 2023a. \emph{Forcats: Tools for Working with Categorical
Variables (Factors)}. \url{https://forcats.tidyverse.org/}.

\bibitem[\citeproctext]{ref-R-stringr}
---------. 2023b. \emph{Stringr: Simple, Consistent Wrappers for Common
String Operations}. \url{https://stringr.tidyverse.org}.

\bibitem[\citeproctext]{ref-R-tidyverse}
---------. 2023c. \emph{Tidyverse: Easily Install and Load the
Tidyverse}. \url{https://tidyverse.tidyverse.org}.

\bibitem[\citeproctext]{ref-tidyverse2019}
Wickham, Hadley, Mara Averick, Jennifer Bryan, Winston Chang, Lucy
D'Agostino McGowan, Romain François, Garrett Grolemund, et al. 2019.
{``Welcome to the {tidyverse}.''} \emph{Journal of Open Source Software}
4 (43): 1686. \url{https://doi.org/10.21105/joss.01686}.

\bibitem[\citeproctext]{ref-R-readxl}
Wickham, Hadley, and Jennifer Bryan. 2025. \emph{Readxl: Read Excel
Files}. \url{https://readxl.tidyverse.org}.

\bibitem[\citeproctext]{ref-R-usethis}
Wickham, Hadley, Jennifer Bryan, Malcolm Barrett, and Andy Teucher.
2024. \emph{Usethis: Automate Package and Project Setup}.
\url{https://usethis.r-lib.org}.

\bibitem[\citeproctext]{ref-R-sessioninfo}
Wickham, Hadley, Winston Chang, Robert Flight, Kirill Müller, and Jim
Hester. 2025. \emph{Sessioninfo: R Session Information}.
\url{https://github.com/r-lib/sessioninfo\#readme}.

\bibitem[\citeproctext]{ref-R-ggplot2}
Wickham, Hadley, Winston Chang, Lionel Henry, Thomas Lin Pedersen,
Kohske Takahashi, Claus Wilke, Kara Woo, Hiroaki Yutani, Dewey
Dunnington, and Teun van den Brand. 2024. \emph{Ggplot2: Create Elegant
Data Visualisations Using the Grammar of Graphics}.
\url{https://ggplot2.tidyverse.org}.

\bibitem[\citeproctext]{ref-R-dplyr}
Wickham, Hadley, Romain François, Lionel Henry, Kirill Müller, and Davis
Vaughan. 2023. \emph{Dplyr: A Grammar of Data Manipulation}.
\url{https://dplyr.tidyverse.org}.

\bibitem[\citeproctext]{ref-R-purrr}
Wickham, Hadley, and Lionel Henry. 2025. \emph{Purrr: Functional
Programming Tools}. \url{https://purrr.tidyverse.org/}.

\bibitem[\citeproctext]{ref-R-readr}
Wickham, Hadley, Jim Hester, and Jennifer Bryan. 2024. \emph{Readr: Read
Rectangular Text Data}. \url{https://readr.tidyverse.org}.

\bibitem[\citeproctext]{ref-R-devtools}
Wickham, Hadley, Jim Hester, Winston Chang, and Jennifer Bryan. 2022.
\emph{Devtools: Tools to Make Developing r Packages Easier}.
\url{https://devtools.r-lib.org/}.

\bibitem[\citeproctext]{ref-R-scales}
Wickham, Hadley, Thomas Lin Pedersen, and Dana Seidel. 2023.
\emph{Scales: Scale Functions for Visualization}.
\url{https://scales.r-lib.org}.

\bibitem[\citeproctext]{ref-R-tidyr}
Wickham, Hadley, Davis Vaughan, and Maximilian Girlich. 2024.
\emph{Tidyr: Tidy Messy Data}. \url{https://tidyr.tidyverse.org}.

\bibitem[\citeproctext]{ref-R-cowplot}
Wilke, Claus O. 2024. \emph{Cowplot: Streamlined Plot Theme and Plot
Annotations for Ggplot2}. \url{https://wilkelab.org/cowplot/}.

\bibitem[\citeproctext]{ref-R-ggtext}
Wilke, Claus O., and Brenton M. Wiernik. 2022. \emph{Ggtext: Improved
Text Rendering Support for Ggplot2}. \url{https://wilkelab.org/ggtext/}.

\bibitem[\citeproctext]{ref-knitr2014}
Xie, Yihui. 2014. {``Knitr: A Comprehensive Tool for Reproducible
Research in {R}.''} In \emph{Implementing Reproducible Computational
Research}, edited by Victoria Stodden, Friedrich Leisch, and Roger D.
Peng. Chapman; Hall/CRC.

\bibitem[\citeproctext]{ref-knitr2015}
---------. 2015. \emph{Dynamic Documents with {R} and Knitr}. 2nd ed.
Boca Raton, Florida: Chapman; Hall/CRC. \url{https://yihui.org/knitr/}.

\bibitem[\citeproctext]{ref-R-knitr}
---------. 2025. \emph{Knitr: A General-Purpose Package for Dynamic
Report Generation in r}. \url{https://yihui.org/knitr/}.

\bibitem[\citeproctext]{ref-rmarkdown2018}
Xie, Yihui, J. J. Allaire, and Garrett Grolemund. 2018. \emph{R
Markdown: The Definitive Guide}. Boca Raton, Florida: Chapman; Hall/CRC.
\url{https://bookdown.org/yihui/rmarkdown}.

\bibitem[\citeproctext]{ref-rmarkdown2020}
Xie, Yihui, Christophe Dervieux, and Emily Riederer. 2020. \emph{R
Markdown Cookbook}. Boca Raton, Florida: Chapman; Hall/CRC.
\url{https://bookdown.org/yihui/rmarkdown-cookbook}.

\bibitem[\citeproctext]{ref-R-kableExtra}
Zhu, Hao. 2024. \emph{kableExtra: Construct Complex Table with Kable and
Pipe Syntax}. \url{http://haozhu233.github.io/kableExtra/}.

\end{CSLReferences}

\end{document}
